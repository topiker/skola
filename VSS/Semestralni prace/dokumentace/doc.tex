\documentclass{article}
\usepackage[czech]{babel}
\usepackage[utf8]{inputenc}
\usepackage{graphicx}
\usepackage{pdfpages}
\usepackage{textgreek}
\usepackage{xargs} 
\usepackage{xcolor}
\usepackage{pdfpages}
\usepackage{cite}
\usepackage[colorinlistoftodos,prependcaption,textsize=tiny]{todonotes}
\newcommandx{\xtodo}[2][1=]{\todo[linecolor=red,backgroundcolor=red!25,bordercolor=red,#1]{#2}}


\usepackage{listings}
\usepackage{color}

\definecolor{dkgreen}{rgb}{0,0.6,0}
\definecolor{gray}{rgb}{0.5,0.5,0.5}
\definecolor{mauve}{rgb}{0.58,0,0.82}

\lstset{frame=tb,
  language=Sql,
  aboveskip=3mm,
  belowskip=3mm,
  showstringspaces=false,
  columns=flexible,
  basicstyle={\small\ttfamily},
  numbers=none,
  numberstyle=\tiny\color{gray},
  keywordstyle=\color{blue},
  commentstyle=\color{dkgreen},
  stringstyle=\color{mauve},
  breaklines=true,
  breakatwhitespace=true,
  tabsize=3
}




\begin{document}


%---------------------------------------------------------------------------------------------------------------------------------------------------------------%


\begin{titlepage}	
	\begin{center}
		\includegraphics[width=5cm]{logo.jpg}\\[3.5cm]
		{\Huge KIV/VSS}\\[0.5cm]
		{\Large Semestrální práce}\\[0.5cm]
		{\large  Miroslav Liška – A17N0081P}\\[0.5cm]
		{\large  topiker@students.zcu.cz}\\[0.5cm]
		{\large   9.12.1992}\\[0.5cm]
		\vfill

		{\large \today}

	\end{center}
\end{titlepage}



\section{Zadání} %%%%%%%%%%%%%%%%%%%%%%%%%%%%%%%%%%%%%%%%%%%%%%%%%%%%%%%%%%%%%%%%%%%%%%%%%%%%%%%%%%%%%%%%%%%%%%%%%%%%%%%%%%%%%%%%%%%%%%%%%%%%%%%%%%%%%%%%%%
\setcounter{page}{1}
Cílem semestrální práce je vytvoření analytického výkonnostního nebo spolehlivostního modelu zadaného systému a jeho řešení. Dále vytvoření simulačního modelu pro tentýž problém a závěřem obě řešení porovnat.

\section{Analýza}

\section{Analytické řešení}

$$
\Lambda_1 = \lambda + (1 - P_{8}) \cdot (\Lambda_{7} + \Lambda_{8}) + (1 - P_{9}) \cdot \Lambda_{9}
$$
$$
\Lambda_{2} = P_1 \cdot \Lambda_{1} 
$$
$$
\Lambda_{3} = P_2 \cdot \Lambda_{1} 
$$
$$
\Lambda_{4} = P_3 \cdot \Lambda_{1} 
$$
$$
\Lambda_{5} = P_4 \cdot \Lambda_{1} 
$$
$$
\Lambda_{6} = P_5 \cdot \Lambda_{1} 
$$
$$
\Lambda_{7} = \Lambda_{2} + \Lambda_{3} + \Lambda_{4}  + \Lambda_{5} + \Lambda_{6} 
$$
$$
\Lambda_{8} = P_6 \cdot \Lambda_{7} 
$$
$$
\Lambda_{9} = P_7 \cdot \Lambda_{7} 
$$
$$
\Lambda_{10} = P_8 \cdot (\Lambda_{8} + \Lambda_{9}) 
$$
$$
\Lambda_{E1} = P_9 \cdot \Lambda_{10}
$$


\section{Řešení}%%%%%%%%%%%%%%%%%%%%%%%%%%%%%%%%%%%%%%%%%%%%%%%%%%%%%%%%%%%%%%%%%%%%%%%%%%%%%%%%%%%%%%%%%%%%%%%%%%%%%%%%%%%%%%%%%%%%%%%%%%%%%%%%%%%%%%%%%%

Na vstupu do uzlu se sčítají frekvence na příchozích hranách. 
Zároveň musí platit, že to, co do uzlu přiteče, musí také odtéci. 
Tento předpoklad platí, pokud je systém ve stacionárním režimu.

Ze zadání vyplývá, že se vstupní proud rozdělí na dva podle zadaných pravděpodobností. 
Rovnice pro výpočet střední frekvence jednotlivých uzlů vypadají následovně:
$$
\Lambda_1 = \lambda T_{S_1} = 4 \cdot 0,8 = 3,2     
$$
$$
\Lambda_2 = \lambda T_{S_2} = 8 \cdot 0,2 = 0,125 
$$

Nyní když známe střední frekvence jednotlivých uzlů, určíme jejich zatížení.

$$
\rho_1 = \Lambda_1\lambda T_{S_1} = 3,2 \cdot 0,25 = 0,8     
$$
$$
\rho_2 = \Lambda_2\lambda T_{S_2} = 0,8 \cdot 0,125 = 0,1
$$

Zatížení jednotlivých uzlů je menší než jedna, můžeme tedy o systému prohlásit, že je ve stacionárním režimu.

Nyní určíme střední počet požadavků pro jednotlivé uzly. K tomu potřebujeme znát zatížení uzlu a počet kanálů obsluhy:

$$
L_{q_1} = \frac{m\rho_{1}}{1-\rho_{1}^m} = \frac{1\cdot0,8}{1-0,8^1} = 4
$$
$$
L_{q_2} = \frac{m\rho_{2}}{1-\rho_{2}^m} = \frac{1\cdot0,1}{1-0,1^1} = 0,111
$$

$$
L_{q} = L_{q_1} + L_{q_2} = 4 + 0,111
$$

Počet kanálů obsluhy je v tomto případě vždy 1.

Pro výpočet střední doby obsluhy v uzlu potřebujeme znát střední doby kanálů obsluhy a zatížení systému:
$$
T_{q_1} =\frac{T_{s_1}}{1 - \rho_{1}^m} = \frac{0,25}{1-0,8} = 1,25
T_{q_2} =\frac{T_{s_2}}{1 - \rho_{2}^m} = \frac{0,125}{1-0,1} = 0.138
$$

Pro výpočet střední doby obsluhy systému celkově se použije vzorec:

$$
T_{q} =\frac{1}{\Lambda_0}L_q = =\frac{1}{\Lambda_0} = \frac{4,111}{4} = 1,02777
T_{q_2} =\frac{T_{s_2}}{1 - \rho_{2}^m} = \frac{0,125}{1-0,1} = 0.138
$$

Pro určení, kolik procent z dlouhého časového intervalu sledování sítě bude fronta 2 prázdná, nahlédneme na frontu 2 jako na markovský model typu M/1/1.
Fronta je prázdná ve stavu 0, kdy nepřišel žádný požadavek, nebo byl z fronty vybrán poslední požadavek. 
Odpovědí na dotaz je tedy ustálená pravděpodobnost toho, že jsme ve stavu 0.
Pro výpočet této pravděpodobnosti jsem využil nástroje MARKOV2. Model jsem definoval tímto scriptem:\\
\\
\newline
\texttt
{
module example [200];\\
\#define size 200\\
\#define lamda 0.8\\
\#define mi 8\\
\newline
for(i ;0; size-2)\{\\
	{[}i{]}->{[i+1]};\\
\}\\
\newline
for(i ;0; size-2)\{\\
	{[}i+1{]}->mi {[i]};\\
\}\\
}

a výsledek získal MMSQL dotazem:\newline
\newline
\texttt
{
load "example" as buf\newline
select p{[0]}*100 \newline
as Prazdno from buf\newline
}
\newline
Výsledkem pak je:
\texttt
\newline
\texttt{
Prazdno \newline
90
}

Fronta 2 bude tedy 90\% času prázdná.

\section{Závěr} %%%%%%%%%%%%%%%%%%%%%%%%%%%%%%%%%%%%%%%%%%%%%%%%%%%%%%%%%%%%%%%%%%%%%%%%%%%%%%%%%%%%%%%%%%%%%%%%%%%%%%%%%%%%%%%%%%%%%%%%%%%%%%%%%%%%%%%%%%
V rámci úkoly byly vypočítány střední frekvence jednotlivých uzlů, jejich zatížení, střední délka fronty a střední doba obsluhy. Následně bylo vypočítáno, kolik procent časového intervalu sledování bude druhá fronta prázdná. Společně s úkolem je navíc odevzdán soubor s daty pro program QNAlyzer pro případnou kontrolu.
\end{document}